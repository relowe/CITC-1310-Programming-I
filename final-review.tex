\documentclass{article}

\title{CITC1310 -- Final Exam Review Questions}
\date{}

\begin{document}
\maketitle

\begin{enumerate}
    \item What is a method?
    \item What are parameters?
    \item What are arguments?
    \item How are arguments passed in Java?
    \item Be able to identify the parts of a method signature.
    \item What is method overloading?
    \item What are the rules about method overloading?
    \item What is scope?
    \item Why do we use methods?
    \item What is an object?  (What two things do objects have?)
    \item What is the difference between static and non-static
        methods?
    \item What are classes? How do they relate to objects?
    \item What is a constructor?
    \item What is the minimal set of constructors that a class should
        provide?
    \item What are the two main design goals in OOP, and what does
        each mean?
    \item What are the access modifiers in Java, and what do each do?
    \item What access modifier should be used for variables?
    \item What access modifier should be used for most constructors
        and methods?
    \item What does the following code create?
    \begin{verbatim}
    PhoneRecord record;
    \end{verbatim}
    \item How do you instantiate an object?
    \item In what area of memory are objects created?
    \item What happens when an object has no references?
    \item Be able to write code for a simple class.
    \item Be able to write code which creates and uses an object.
    \item What is the purpose and syntax of each of the following UML
        diagrams:
        \begin{itemize}
            \item Class Diagram
            \item Object Diagram
        \end{itemize}
    \item What is aggregation and composition? Be able to provide an
        example of each and/or identify which is at play.
    \item How is aggregation represented in a UML diagram?
    \item How is composition represented in a UML diagram?
    \item What are the steps in the object oriented analysis and
        design process?
    \item Given a sample problem, be able to carry out the OOAD and
        draw appropriate UML diagrams.
    \item How do you declare arrays?
    \item How do you instantiate arrays?
    \item Is an array an object?
    \item What is the minimum index in an array?
    \item What is the maximum index of an array?
    \item What is a foreach loop used for?  What is its syntax like?
    \item Be able to write a foreach loop.
\end{enumerate}

\end{document}
